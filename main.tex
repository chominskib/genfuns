\documentclass[shortabstract]{imthesis}

\usepackage[utf8]{inputenc}
\usepackage{amsmath}
\usepackage{amsthm}
\usepackage{amsfonts}
\usepackage{relsize}
\usepackage{mathtools}
\usetikzlibrary{calc,arrows.meta,positioning,automata}
\usetikzlibrary{patterns,decorations.pathreplacing}

\polishtitle    {Funkcje tworzące}
\englishtitle   {Generating functions}
\polishabstract {\ldots}
\englishabstract{\ldots}
\author         {Bartosz Chomiński}
\advisor        {prof. dr hab. Dariusz Buraczewski}
\date           {PROJEKT}
% Dane do oswiadczenia o autorskim wykonaniu
%\transcriptnum {}                     % Numer indeksu
%\advisorgen    {dr. Jana Kowalskiego} % Nazwisko promotora w dopelniaczu
%%%%%

\begin{document}

%%%%% POCZĄTEK ZASADNICZEGO TEKSTU PRACY

\chapter{Szkic struktury pracy}

\section{Przedmowa}

Celem jest napisanie wysokiej jakości materiału dydaktycznego celującego w ambitnego licealistę zaznajomionego z podstawowymi technikami używanymi na OM, który uczy czytelnika podstawowego warsztatu i zastosowań funkcji tworzących, wykładniczych funkcji tworzących i szeregów Dirichleta, a ponadto przekonuje go, że ta wiedza będzie przydatna.

\section{Motywacja}

Tu opowiem skąd bierze się potrzeba używania tego narzędzia. Najlepiej chyba jest to zrobić przedstawiając zadanie, które trudno rozwiązać inaczej, niż tą teorią.

\section{Podstawy teorii}

W każdej sekcji celem jest przedstawienie teorii do poziomu prostych lematów typu "$\mu * \mathbf{1} = \mathrm{id}$".

\subsection{Funkcje tworzące}

Definicje formalne, tłumaczenie kilku prostych ciągów nieskończonych na ich funkcje tworzące, podstawowy lemat o jednoznaczności zapisu, tłumaczenie działań na ciągach na działania na funkcjach, splot Cauchy'ego.

\subsection{Wykładnicze funkcje tworzące}

Definicje formalne, tłumaczenie kilku prostych ciągów nieskończonych na ich funkcje tworzące, podstawowy lemat o jednoznaczności zapisu, tłumaczenie działań na ciągach na działania na funkcjach.

\subsection{Szeregi Dirichleta}

Definicje formalne, tłumaczenie kilku prostych ciągów nieskończonych na ich szeregi Dirichleta, podstawowy lemat o jednoznaczności zapisu, tłumaczenie działań na ciągach na działania na szeregach, funkcje multiplikatywne, splot Dirichleta, algebra splotu.

\subsection{Podsumowanie porównawcze}

Tabela o czterech kolumnach: działanie na ciągach, działanie na funkcjach tworzących, działanie na wykładniczych funkcjach tworzących, działanie na szeregach Dirichleta.

Wśród działań: dodawanie, mnożenie przez skalar, translacja, splot Cauchy'ego, mnożenie przez konkretne ciągi, etc.

\section{Zastosowania}

Tu opowiem o kilku przykładach zastosowań funkcji tworzących: rozwiązywanie rekurencji liniowych, wzór ogólny na liczby Catalana, jedno-dwa zadania z OM, funkcje charakterystyczne zmiennych losowych.

\section{Zaawansowane metody}

Tu opowiem o kilku ciekawszych metodach związanych z funkcjami tworzącymi: roots of unity filters, snake oil method, formuła inwersyjna Lagrange'a.

\section{Zadania do samodzielnego rozwiązania}

Garstka około 20 zadań typu olimpijskiego i kilku wykraczających poza szkołę średnią, które dają się efektownie rozwiązać narzędziami opisanymi powyżej.

\section{Rozwiązania zadań}

Rozsądnej jakości i dokładności rozwiązania do powyższych zadań.

\section{Podziękowania}

	Autor podziękuje tu kilku osobom, które spotkał w czasie studiowania.

\ldots

%%%%% BIBLIOGRAFIA

\bibliographystyle{plain}
\bibliography{refs}

\end{document}
